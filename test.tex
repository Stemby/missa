% test.tex 
\documentclass[leaflet,rite=ambrosian,litcolor=red]{missa}
\usepackage[utf8x]{inputenc}
\usepackage[italian]{babel}
\usepackage[T1]{fontenc}

\usepackage{lipsum}

%\usepackage[short]{datetime}

\title{La Messa}
\author{Carlo Stemberger}

\begin{document}
%\maketitle

%\dayofweekname{31}{10}{2002}

\section{Liturgia vigiliare vespertina}
\lipsum[1]

\section{Riti di introduzione}
\subsection{All'ingresso}

\subsection{Atto penitenziale}
\priest{Fratelli carissimi, con fiducia disponiamoci al pentimento e riconosciamo i nostri peccati, perché il Signore, nella sua infinita misericordia, ci doni di partecipare a questa celebrazione con una coscienza pura e un cuore lieto e operoso.}

\silence

\priest[Don Albino]{Tu che ti sei fatto povero per arricchirci: \kyrie}

\all{\kyrie}

\priest{Tu che sei il difensore dei poveri e il rifugio dei deboli: \kyrie}

\all{\kyrie}

\priest{Tu che perdoni molto a chi molto ama: \kyrie}

\all{\kyrie}

\priest{Dio onnipotente abbia misericordia di noi, perdoni i nostri peccati e ci conduca alla vita eterna.} \hfill
\all{\amen}

\subsection{Gloria}
\subsection{inizio dell'assemblea liturgica}
\priest{Preghiamo.\hfill\silence \\ O Dio, che nel tuo ineffabile amore hai creato l'universo, donaci di adorarti sempre con tutto il nostro essere e di amare ogni uomo con affetto giusto e fraterno. Per Gesù~Cristo\dots}

\all{\amen}

\section{Liturgia della parola}
\subsection{Lettura}

\subsection{Salmo}
\all{Ascolta, Signore, il povero che t'invoca.}
\lilypondfile{lily/testlily}
\reader{Porgi l'orecchio, Signore, alle mie parole: intendi il mio lamento. Sii attento alla voce del mio grido, o mio re e mio Dio, perché a te, Signore, rivolgo la mia preghiera.}~\R

\reader{Tu non sei un Dio che gode del male, non è tuo ospite il malvagio; gli stolti non resistono al tuo sguardo.}~\R

\end{document}
