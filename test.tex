% test.tex 
\documentclass[leaflet,rite=ambrosian,litcolor=red,biblerefstyle=CEI]{missa}
\usepackage[utf8x]{inputenc}
\usepackage[italian]{babel}
\usepackage[T1]{fontenc}

\usepackage{lipsum}

%\usepackage[short]{datetime}

\newcommand{\TODO}{\makecolorbox{green}{\textcolor{red}{TODO}}}

\title{La Messa}
\author{Carlo Stemberger}
\date{12 settembre 2010} % use datetime instead

\begin{document}
%\maketitle

%\dayofweekname{31}{10}{2002}

\section{Liturgia vigiliare vespertina}
\subsection{Vangelo della Risurrezione}
\lipsum[1]

\section{Riti di introduzione}
\subsection{All'ingresso}\TODO \reference{\cantemus{8}}\par
\R \hfill \TODO

Quando tu ci chiami, o Signore,\\
gioisce i cuore se tu ci parli.\\
Oggi tu ci inviti alla tua mensa\\
e noi cantiamo a te, o Signore.

Benedirò il Signore ora e sempre, in ogni tempo: sulla mia bocca sempre avrò la sua lode. \R

Gustate e vedete quanto è buono il Signore: felice l'uomo che in lui si rifugia. \R

\otherwise \hfill \reference{\bibleverse{Psalms}(118:137,124)}

\all{Tu sei giusto, Signore, e retto nei tuoi giudizi; usa misericordia col tuo servo.}

\subsection{Atto penitenziale}
\priest{Fratelli e sorelle, invitati dal Signore alla sua mensa di salvezza, con fiducia, riconosciamoci tutti peccatori e, perdonandoci a vicenda dal profondo del cuore, invochiamo la misericordia di Dio.}

\silence

\priest{Tu, Figlio di Dio, che sei stato annunciato dai profeti e atteso dai giusti: \kyrie}

\all{\kyrie}

\priest{Tu, Figlio dell'uomo, che hai posto le radici nel popolo dell'alleanza e ci fai eredi dell'antica promessa: \kyrie}

\all{\kyrie}

\priest{Tu, Figlio unigenito del Padre, che giustifichi nella fede che opera per mezzo della carità: \kyrie}

\all{\kyrie}

\priest{Dio onnipotente abbia misericordia di noi, perdoni i nostri peccati e ci conduca alla vita eterna.} \hfill
\all{\amen}

\subsection{Gloria}\TODO \reference{\cantemus{24}}\par
\all{\gloria}

\subsection{inizio dell'assemblea liturgica}
\priest{Preghiamo.\hfill\silence \\ Serbaci nella tua fedeltà, o Dio vivo e vero, e conforta con la tua grazia i nostri cuori; donaci di attendere con gioiosa dedizione al canto della tua lode e di crescere nell'amore fraterno. Per Gesù~Cristo, %\dots}
tuo Figlio, nostro Signore e nostro Dio, che vive e regna con te, nell'unità dello Spirito Santo, per tutti i secoli dei secoli.} \hfill
\all{\amen}

\section{Liturgia della Parola}
\subsection{Lettura}\TODO \reference{\bibleverse{Isaiah}(5:1-7)}\par
\TODO\par\TODO\par
\lipsum[3]
\twotl
%\hfill \all{\tbtg}
\par \all{\tbtg}

\subsection{Salmo}\TODO \reference{\bibleverse{Psalms}(79:)~(80)}\par % TODO: verify psalms syntax
\all{La vigna del Signore è il suo popolo.}
\par\null
\lilypondfile{lily/testlily}
\null\par
\lector{Hai sradicato una vite dell'Egitto, hai scacciato le genti e l'hai trapiantata. Ha esteso i suoi tralci fino al mare, e arrivavano al fiume i suoi germogli.}~\R

\lector{Dio degli eserciti, ritorna! Guarda dal cielo e vedi e visita questa vigna, proteggi quello che la tua destra ha piantato, il figlio dell'uomo che per te hai reso forte.}~\R

\lector{Da te mai più ci allontaneremo, facci rivivere e noi invocheremo il tuo nome. Signore, Dio degli eserciti, fa' che ritorniamo, fa' splendere il tuo volto e noi saremo salvi.}~\R

\subsection{Epistola}\TODO \reference{\bibleverse{Galatians}(2:15-20)}\par
\TODO\par\TODO\par
\lipsum[4]
\twotl% \hfill \all{\tbtg}

\all{\tbtg}

\subsection{Canto al Vangelo}\TODO \reference{Cf~\bibleverse{Matthew}(8:11)~(\cantemus{40})}\par

\all{\alleluia}

\lector{Molti verranno dall'oriente e dall'occidente, dice il Signore, e siederanno a mensa nel regno dei cieli.}~\R

\subsection{Vangelo}\TODO \reference{\bibleverse{Matthew}(21:28-32)}\par
\TODO\par\TODO\par\TODO\par
\all{\gtyl}

\lipsum[5]
\tgotl% \hfill \all{\ptyljc}

\all{\ptyljc}

\subsection{Dopo il Vangelo}\TODO \reference{\cantemus{502}}\par
Ricorda la promessa fatta al tuo servo,\\
con la quale mi hai dato speranza.\\
Questo mi consola nella mia miseria;\\
la tua Parola mi dona la vita.

\otherwise \hfill \reference{\bibleverse{Psalms}(101:12-13)}

\all{I miei giorni sono come ombra che declina, come l'erba tagliata inaridisco. Ma tu, Signore, rimani in eterno, il tuo ricordo per ogni generazione.}

\subsection{Preghiera universale}
\priest{Fratelli e sorelle, «il Signore è giusto e retto nei suoi giudizi»: confortati dalla sua bontà e misericordia, presentiamo a lui le nostre preghiere.}

\all{Vieni, Signore, visita la tua vigna!}

\lector{Per la Chiesa, perché sia segno visibile ed efficace dell'azione e della cura di Dio per ogni uomo: preghiamo.}~\R

\lector{Per i governanti, perché difendano la dignità della persona e, con fermezza, si oppongano a ogni forma di ingiustizia e di sopruso: preghiamo.}~\R

\lector{Per noi, perché, riconoscendo i doni del Signore, sappiamo crescere ogni giorno portando frutti di amore, di fraternità e di pace: preghiamo.}~\R

\petitions

\subsection{Conclusione liturgia della Parola}
\priest{Non abbandonarci, o Dio, e non privarci dei tuoi doni di grazia; venga dalla pietà del tuo cuore di padre quanto non è dato di compiere alla nostra debolezza. Per Cristo nostro Signore.} \hfill \all{\amen}

\section{Liturgia eucaristica}
\subsection{Professione di fede}
\all{\creed}

\subsection{Sui doni}
\priest{Accogli, o Dio, le nostre offerte in questo incontro mirabile della nostra povertà e della tua grandezza: noi ti presentiamo le cose che da te ci provengono, tu donaci in cambio te stesso. Per Cristo nostro Signore.} \hfill \all{\amen}

\subsection{Prefazio}
\priest{È veramente cosa buona e giusta esaltarti, Dio di misericordia infinita. Cristo Signore nostro, nascendo dalla Vergine, ci ha liberato dall'antica decadenza e ha rinnovato la nostra natura mortale. Con la sua passione ha espiato le nostre colpe, con la sua risurrezione ci ha aperto il varco alla vita eterna e con la sua ascensione alla tua gloria, o Padre, ci ha dischiuso le porte del regno. Per questo disegno di grazia, uniti a tutte le voci adoranti del cielo e della terra, eleviamo a te, unico e immenso Dio col Figlio e con lo Spirito Santo, l'inno della triplice lode:}

%\all{\sanctus}
\all{Santo\dots}
\hfill \reference{\cantemus{99}}

\subsection{Anamnesi}\TODO \reference{\cantemus{89}}\par
\priest{Mistero della fede.}

\all{Annunziamo la tua morte, Signore, proclamiamo la tua risurrezione, nell'attesa della tua venuta.}

\subsection{Allo spezzare del pane}\TODO \reference{\cantemus{588}}\par
Vino, versato ai discepoli,\\
sangue di un Dio crocifisso,\\
in te la nostra gioia, in te!\\
in te la nostra gioia.

\otherwise

\all{Il Pane di vita è spezzato, il Calice è benedetto. Il tuo corpo ci nutra, o Dio nostro, il tuo sangue ci dia vita e ci salvi.}

\subsection{Alla Comunione}
Pane di vita nuova,\\
vero cibo dato agli uomini,\\
nutrimento che sostiene il mondo,\\
dono splendido di grazia.

\R \hfill \TODO\par
Pane della vita,\\
sangue di salvezza,\\
vero corpo, vera bevanda,\\
cibo di grazia per il mondo.

Tu sei sublime frutto\\
di quell'albero di vita\\
che Adamo non poté toccare:\\
ora è in Cristo a noi donato.~\R

Sei l'Agnello immolato\\
nel cui sangue è la salvezza,\\
memoriale della vera pasqua\\
della nuova Alleanza.~\R

\otherwise: \hfill \reference{\bibleverse{Psalms}(103:33-34)}

\all{Finché avrò vita canterò al Signore, finché esisto, voglio inneggiare a Dio. A lui sia gradito il mio canto, in lui sarà la mia gioia.}

\subsection{Dopo la Comunione}
\priest{Preghiamo.\hfill\silence \\ O Dio, che alla tua mensa ci hai nutrito col Pane del cielo, fa' che questo divino alimento ravvivi in noi l'amore per te e ci spinga a vederti e a servirti nei nostri fratelli. Per Cristo nostro Signore.}

\all{\amen}

\end{document}
