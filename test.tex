% test.tex 
\documentclass[leaflet,rite=ambrosian,litcolor=red,biblerefstyle=CEI]{missa}
\usepackage[utf8x]{inputenc}
\usepackage[italian]{babel}
\usepackage[T1]{fontenc}

\usepackage{lipsum}

%\usepackage[short]{datetime}

\title{La Messa}
\author{Carlo Stemberger}

\begin{document}
%\maketitle

%\dayofweekname{31}{10}{2002}

\section{Liturgia vigiliare vespertina}
\subsection{Vangelo della Risurrezione}
\lipsum[1]

\section{Riti di introduzione}
\subsection{All'ingresso}
\lipsum[2]
\otherwise \hfill \reference{\bibleverse{Psalms}(24:16,18)}

\all{Volgi il tuo sguardo misericordioso sopra di me, Signore, perché sono povero e solo. Vedi che sono oppresso e travagliato, perdona i miei peccati.}
\subsection{Atto penitenziale}
\priest{Fratelli carissimi, con fiducia disponiamoci al pentimento e riconosciamo i nostri peccati, perché il Signore, nella sua infinita misericordia, ci doni di partecipare a questa celebrazione con una coscienza pura e un cuore lieto e operoso.}

\silence

\priest{Tu che ti sei fatto povero per arricchirci: \kyrie}

\all{\kyrie}

\priest{Tu che sei il difensore dei poveri e il rifugio dei deboli: \kyrie}

\all{\kyrie}

\priest{Tu che perdoni molto a chi molto ama: \kyrie}

\all{\kyrie}

\priest{Dio onnipotente abbia misericordia di noi, perdoni i nostri peccati e ci conduca alla vita eterna.} \hfill
\all{\amen}

\subsection{Gloria}
\subsection{inizio dell'assemblea liturgica}
\priest{Preghiamo.\hfill\silence \\ O Dio, che nel tuo ineffabile amore hai creato l'universo, donaci di adorarti sempre con tutto il nostro essere e di amare ogni uomo con affetto giusto e fraterno. Per Gesù~Cristo\dots}

\all{\amen}

\section{Liturgia della Parola}
\subsection{Lettura}
\lipsum[3]

\subsection{Salmo}
\all{Ascolta, Signore, il povero che t'invoca.}
\par\null
\lilypondfile{lily/testlily}
\null\par
\lector{Porgi l'orecchio, Signore, alle mie parole: intendi il mio lamento. Sii attento alla voce del mio grido, o mio re e mio Dio, perché a te, Signore, rivolgo la mia preghiera.}~\R

\lector{Tu non sei un Dio che gode del male, non è tuo ospite il malvagio; gli stolti non resistono al tuo sguardo.}~\R

\lector{Tu hai in odio tutti i malfattori, tu distruggi chi dice menzogne. Sanguinari e ingannatori, il Signore li detesta.}~\R

\subsection{Epistola}

\subsection{Canto al Vangelo}
\all{\alleluia}

\lector{Beati coloro che custodiscono la parola di Dio con cuore integro e buono e producono frutto con perseveranza.}~\R

\subsection{Vangelo}

\subsection{Dopo il Vangelo}

\subsection{Preghiera universale}
\priest{Fratelli e sorelle, animati da carità sincera, eleviamo la nostra unanime preghiera a Dio, principio di ogni bontà e bellezza.}

\all{Ascoltaci, Signore.}

\lector{Per la Chiesa, perché si allontani da ciò che è fuggevole e vano e, con libertà, orienti il proprio cuore ai beni eterni: preghiamo.}~\R

\lector{Per l'umanità intera, perché, abbandonando l'egoismo e l'indifferenza del mondo, percorra la via della giustizia, del dialogo e della solidarietà: preghiamo.}~\R

\lector{Per ciascuno di noi, perché, nell'amore sincero per Dio e i fratelli, trovi quanto è essenziale per la vita: preghiamo.}~\R

\subsection{Conclusione liturgia della Parola}
\priest{La tua grazia, o Dio onnipotente, ci protegga e ci serbi nel tuo servizio; e, poiché senza di te non possiamo operare secondo giustizia, donaci tu di piacerti in tutta la nostra vita. Per Cristo nostro Signore.} \hfill
\all{\amen}

\section{Liturgia eucaristica}
\subsection{Professione di fede}
\all{\creed}

\subsection{Sui doni}
\priest{Accogli, o Padre, l'offerta del tuo popolo; esaudisci la fiduciosa preghiera e santifica i nostri giorni. Per Cristo nostro Signore.} \hfill
\all{\amen}

\subsection{Prefazio}
\priest{È veramente cosa buona e giusta renderti grazie, o Dio di infinita potenza. Tu doni alla Chiesa di Cristo di celebrare misteri ineffabili nei quali la nostra esiguità di creature mortali si insublima in un rapporto eterno, e la nostra esistenza nel tempo comincia a fiorire nella vita senza fine. Così, seguendo il tuo disegno d'amore, l'uomo trascorre da una condizione di morte a una prodigiosa salvezza. Ammiràti e felici, noi ci uniamo al coro di tutte le voci che in terra e in cielo cantano la tua gloria e nella comune letizia eleviamo l'inno di lode:}% \hfill
%\all{Santo\dots}

\all{\sanctus} \hfill \reference{\cantemus{76}}

\subsection{Anamnesi}
\priest{Mistero della fede.}

\all{Annunziamo la tua morte, Signore, proclamiamo la tua risurrezione, nell'attesa della tua venuta.}

\subsection{Allo spezzare del pane}
\otherwise

\all{Fa' splendere il tuo volto sul tuo servo e salvami, per tua misericordia. Che io non resti confuso, Signore, perché ti ho invocato.}

\subsection{Alla Comunione}
\subsection{Dopo la Comunione}
\priest{Preghiamo.\hfill\silence \\ O Dio, che ci hai resi partecipi dell'unico Pane e dell'unico Calice, fa' che portiamo frutti di vita eterna per la salvezza del mondo, poi che ci concedi la gioia di essere una sola cosa in Cristo Signore, che vive e regna nei secoli dei secoli.}

\all{\amen}

\end{document}
